%% Generated by Sphinx.
\def\sphinxdocclass{report}
\documentclass[letterpaper,10pt,english]{sphinxmanual}
\ifdefined\pdfpxdimen
   \let\sphinxpxdimen\pdfpxdimen\else\newdimen\sphinxpxdimen
\fi \sphinxpxdimen=.75bp\relax
\ifdefined\pdfimageresolution
    \pdfimageresolution= \numexpr \dimexpr1in\relax/\sphinxpxdimen\relax
\fi
%% let collapsible pdf bookmarks panel have high depth per default
\PassOptionsToPackage{bookmarksdepth=5}{hyperref}

\PassOptionsToPackage{booktabs}{sphinx}
\PassOptionsToPackage{colorrows}{sphinx}

\PassOptionsToPackage{warn}{textcomp}
\usepackage[utf8]{inputenc}
\ifdefined\DeclareUnicodeCharacter
% support both utf8 and utf8x syntaxes
  \ifdefined\DeclareUnicodeCharacterAsOptional
    \def\sphinxDUC#1{\DeclareUnicodeCharacter{"#1}}
  \else
    \let\sphinxDUC\DeclareUnicodeCharacter
  \fi
  \sphinxDUC{00A0}{\nobreakspace}
  \sphinxDUC{2500}{\sphinxunichar{2500}}
  \sphinxDUC{2502}{\sphinxunichar{2502}}
  \sphinxDUC{2514}{\sphinxunichar{2514}}
  \sphinxDUC{251C}{\sphinxunichar{251C}}
  \sphinxDUC{2572}{\textbackslash}
\fi
\usepackage{cmap}
\usepackage[T1]{fontenc}
\usepackage{amsmath,amssymb,amstext}
\usepackage{babel}



\usepackage{tgtermes}
\usepackage{tgheros}
\renewcommand{\ttdefault}{txtt}



\usepackage[Bjarne]{fncychap}
\usepackage{sphinx}

\fvset{fontsize=auto}
\usepackage{geometry}


% Include hyperref last.
\usepackage{hyperref}
% Fix anchor placement for figures with captions.
\usepackage{hypcap}% it must be loaded after hyperref.
% Set up styles of URL: it should be placed after hyperref.
\urlstyle{same}


\usepackage{sphinxmessages}
\setcounter{tocdepth}{3}
\setcounter{secnumdepth}{3}


\title{Inpainting}
\date{Apr 22, 2025}
\release{}
\author{Srini Raghunathan}
\newcommand{\sphinxlogo}{\vbox{}}
\renewcommand{\releasename}{}
\makeindex
\begin{document}

\ifdefined\shorthandoff
  \ifnum\catcode`\=\string=\active\shorthandoff{=}\fi
  \ifnum\catcode`\"=\active\shorthandoff{"}\fi
\fi

\pagestyle{empty}
\sphinxmaketitle
\pagestyle{plain}
\sphinxtableofcontents
\pagestyle{normal}
\phantomsection\label{\detokenize{index::doc}}


\sphinxstepscope


\chapter{Flatsky module}
\label{\detokenize{flatsky:module-flatsky}}\label{\detokenize{flatsky:flatsky-module}}\label{\detokenize{flatsky::doc}}\index{module@\spxentry{module}!flatsky@\spxentry{flatsky}}\index{flatsky@\spxentry{flatsky}!module@\spxentry{module}}\index{cl2map() (in module flatsky)@\spxentry{cl2map()}\spxextra{in module flatsky}}

\begin{fulllineitems}
\phantomsection\label{\detokenize{flatsky:flatsky.cl2map}}
\pysigstartsignatures
\pysiglinewithargsret{\sphinxcode{\sphinxupquote{flatsky.}}\sphinxbfcode{\sphinxupquote{cl2map}}}{\sphinxparam{\DUrole{n}{flatskymapparams}}\sphinxparamcomma \sphinxparam{\DUrole{n}{cl}}\sphinxparamcomma \sphinxparam{\DUrole{n}{el}\DUrole{o}{=}\DUrole{default_value}{None}}}{}
\pysigstopsignatures
\sphinxAtStartPar
cl2map module \sphinxhyphen{} creates a flat sky map based on the flatskymap parameters and the input power spectra.
Look into make\_gaussian\_realisation for a more general code.
\begin{quote}\begin{description}
\sphinxlineitem{Parameters}\begin{itemize}
\item {} 
\sphinxAtStartPar
\sphinxstyleliteralstrong{\sphinxupquote{flatskymyapparams}} (\sphinxstyleliteralemphasis{\sphinxupquote{list}}) \textendash{} {[}nx, ny, dx, dy{]} where ny, nx = flatskymap.shape; and dy, dx are the pixel resolution in arcminutes.
for example: {[}100, 100, 0.5, 0.5{]} is a 50’ x 50’ flatskymap that has dimensions 100 x 100 with dx = dy = 0.5 arcminutes.

\item {} 
\sphinxAtStartPar
\sphinxstyleliteralstrong{\sphinxupquote{cl}} (\sphinxstyleliteralemphasis{\sphinxupquote{array}}) \textendash{} 1d vector of Cl power spectra: temp / pol. power spectra

\item {} 
\sphinxAtStartPar
\sphinxstyleliteralstrong{\sphinxupquote{el}} (\sphinxstyleliteralemphasis{\sphinxupquote{array}}\sphinxstyleliteralemphasis{\sphinxupquote{ (}}\sphinxstyleliteralemphasis{\sphinxupquote{optional}}\sphinxstyleliteralemphasis{\sphinxupquote{)}}) \textendash{} Multipole over which the signal / noise spectra are defined.
Default is None and el will be np.arange( len(cl\_signal) )

\end{itemize}

\sphinxlineitem{Returns}
\sphinxAtStartPar
\sphinxstylestrong{flatskymap} \textendash{} flatskymap with the given underlying power spectrum cl.

\sphinxlineitem{Return type}
\sphinxAtStartPar
array

\end{description}\end{quote}


\begin{sphinxseealso}{See also:}

\sphinxAtStartPar
{\hyperref[\detokenize{flatsky:flatsky.make_gaussian_realisation}]{\sphinxcrossref{\sphinxcode{\sphinxupquote{make\_gaussian\_realisation}}}}}


\end{sphinxseealso}


\end{fulllineitems}

\index{cl\_to\_cl2d() (in module flatsky)@\spxentry{cl\_to\_cl2d()}\spxextra{in module flatsky}}

\begin{fulllineitems}
\phantomsection\label{\detokenize{flatsky:flatsky.cl_to_cl2d}}
\pysigstartsignatures
\pysiglinewithargsret{\sphinxcode{\sphinxupquote{flatsky.}}\sphinxbfcode{\sphinxupquote{cl\_to\_cl2d}}}{\sphinxparam{\DUrole{n}{el}}\sphinxparamcomma \sphinxparam{\DUrole{n}{cl}}\sphinxparamcomma \sphinxparam{\DUrole{n}{flatskymapparams}}\sphinxparamcomma \sphinxparam{\DUrole{n}{left}\DUrole{o}{=}\DUrole{default_value}{0.0}}\sphinxparamcomma \sphinxparam{\DUrole{n}{right}\DUrole{o}{=}\DUrole{default_value}{0.0}}}{}
\pysigstopsignatures
\sphinxAtStartPar
Interpolating a 1d power spectrum (cl) defined on multipoles (el) to 2D assuming azimuthal symmetry (i.e:) isotropy.
\begin{quote}\begin{description}
\sphinxlineitem{Parameters}\begin{itemize}
\item {} 
\sphinxAtStartPar
\sphinxstyleliteralstrong{\sphinxupquote{el}} (\sphinxstyleliteralemphasis{\sphinxupquote{array}}) \textendash{} Multipoles over which the power spectrium is defined.

\item {} 
\sphinxAtStartPar
\sphinxstyleliteralstrong{\sphinxupquote{cl}} (\sphinxstyleliteralemphasis{\sphinxupquote{array}}) \textendash{} 1d power spectrum that needs to be interpolated on the 2D grid.

\item {} 
\sphinxAtStartPar
\sphinxstyleliteralstrong{\sphinxupquote{flatskymyapparams}} (\sphinxstyleliteralemphasis{\sphinxupquote{list}}) \textendash{} {[}nx, ny, dx, dy{]} where ny, nx = flatskymap.shape; and dy, dx are the pixel resolution in arcminutes.
for example: {[}100, 100, 0.5, 0.5{]} is a 50’ x 50’ flatskymap that has dimensions 100 x 100 with dx = dy = 0.5 arcminutes.

\item {} 
\sphinxAtStartPar
\sphinxstyleliteralstrong{\sphinxupquote{left}} (\sphinxstyleliteralemphasis{\sphinxupquote{float}}) \textendash{} value to be used for interpolation outside of the range (lower side).
default is zero.

\item {} 
\sphinxAtStartPar
\sphinxstyleliteralstrong{\sphinxupquote{right}} (\sphinxstyleliteralemphasis{\sphinxupquote{float}}) \textendash{} value to be used for interpolation outside of the range (higher side).
default is zero.

\end{itemize}

\sphinxlineitem{Returns}
\sphinxAtStartPar
\sphinxstylestrong{cl2d} \textendash{} interpolated power spectrum on the 2D grid.

\sphinxlineitem{Return type}
\sphinxAtStartPar
array, shape is (ny, nx).

\end{description}\end{quote}

\end{fulllineitems}

\index{convert\_eb\_qu() (in module flatsky)@\spxentry{convert\_eb\_qu()}\spxextra{in module flatsky}}

\begin{fulllineitems}
\phantomsection\label{\detokenize{flatsky:flatsky.convert_eb_qu}}
\pysigstartsignatures
\pysiglinewithargsret{\sphinxcode{\sphinxupquote{flatsky.}}\sphinxbfcode{\sphinxupquote{convert\_eb\_qu}}}{\sphinxparam{\DUrole{n}{map1}}\sphinxparamcomma \sphinxparam{\DUrole{n}{map2}}\sphinxparamcomma \sphinxparam{\DUrole{n}{flatskymapparams}}\sphinxparamcomma \sphinxparam{\DUrole{n}{eb\_to\_qu}\DUrole{o}{=}\DUrole{default_value}{1}}}{}
\pysigstopsignatures
\sphinxAtStartPar
Convert EB/QU into each other.
\begin{quote}\begin{description}
\sphinxlineitem{Parameters}\begin{itemize}
\item {} 
\sphinxAtStartPar
\sphinxstyleliteralstrong{\sphinxupquote{map1}} (\sphinxstyleliteralemphasis{\sphinxupquote{array}}) \textendash{} flatsky map of E or Q.

\item {} 
\sphinxAtStartPar
\sphinxstyleliteralstrong{\sphinxupquote{map2}} (\sphinxstyleliteralemphasis{\sphinxupquote{array}}) \textendash{} flatsky map of B or U.

\item {} 
\sphinxAtStartPar
\sphinxstyleliteralstrong{\sphinxupquote{flatskymyapparams}} (\sphinxstyleliteralemphasis{\sphinxupquote{list}}) \textendash{} {[}nx, ny, dx, dy{]} where ny, nx = flatskymap.shape; and dy, dx are the pixel resolution in arcminutes.
for example: {[}100, 100, 0.5, 0.5{]} is a 50’ x 50’ flatskymap that has dimensions 100 x 100 with dx = dy = 0.5 arcminutes.

\item {} 
\sphinxAtStartPar
\sphinxstyleliteralstrong{\sphinxupquote{eb\_to\_qu}} (\sphinxstyleliteralemphasis{\sphinxupquote{bool}}) \textendash{} Either EB\textendash{}\textgreater{}QU or QU\textendash{}\textgreater{}EB.
Default is EB\textendash{}\textgreater{}QU.

\end{itemize}

\sphinxlineitem{Returns}
\sphinxAtStartPar
\begin{itemize}
\item {} 
\sphinxAtStartPar
\sphinxstylestrong{map1\_mod} (\sphinxstyleemphasis{array}) \textendash{} flatsky map of E or Q.

\item {} 
\sphinxAtStartPar
\sphinxstylestrong{map2\_mod} (\sphinxstyleemphasis{array}) \textendash{} flatsky map of B or U.

\end{itemize}


\end{description}\end{quote}

\end{fulllineitems}

\index{get\_lpf\_hpf() (in module flatsky)@\spxentry{get\_lpf\_hpf()}\spxextra{in module flatsky}}

\begin{fulllineitems}
\phantomsection\label{\detokenize{flatsky:flatsky.get_lpf_hpf}}
\pysigstartsignatures
\pysiglinewithargsret{\sphinxcode{\sphinxupquote{flatsky.}}\sphinxbfcode{\sphinxupquote{get\_lpf\_hpf}}}{\sphinxparam{\DUrole{n}{flatskymapparams}}\sphinxparamcomma \sphinxparam{\DUrole{n}{lmin\_lmax}}\sphinxparamcomma \sphinxparam{\DUrole{n}{filter\_type}\DUrole{o}{=}\DUrole{default_value}{0}}}{}
\pysigstopsignatures
\sphinxAtStartPar
Get 2D Fourier filters. Supports low\sphinxhyphen{}pass (LPF), high\sphinxhyphen{}pass (HPF), and band\sphinxhyphen{}pass (BPF) filters.
\begin{quote}\begin{description}
\sphinxlineitem{Parameters}\begin{itemize}
\item {} 
\sphinxAtStartPar
\sphinxstyleliteralstrong{\sphinxupquote{flatskymyapparams}} (\sphinxstyleliteralemphasis{\sphinxupquote{list}}) \textendash{} {[}nx, ny, dx, dy{]} where ny, nx = flatskymap.shape; and dy, dx are the pixel resolution in arcminutes.
for example: {[}100, 100, 0.5, 0.5{]} is a 50’ x 50’ flatskymap that has dimensions 100 x 100 with dx = dy = 0.5 arcminutes.

\item {} 
\sphinxAtStartPar
\sphinxstyleliteralstrong{\sphinxupquote{lmin\_lmax}} (\sphinxstyleliteralemphasis{\sphinxupquote{list}}) \textendash{} Contains lmin and lmax values for the filters.
For low\sphinxhyphen{}pass (LPF), lmax = lmin\_lmax{[}0{]}.
For high\sphinxhyphen{}pass (HPF), lmin = lmin\_lmax{[}0{]}.
For band\sphinxhyphen{}pass (BPF), lmin, lmax = lmin\_lmax.

\item {} 
\sphinxAtStartPar
\sphinxstyleliteralstrong{\sphinxupquote{filter\_type}} (\sphinxstyleliteralemphasis{\sphinxupquote{int}}) \textendash{} 0: LPF
1: HPF
2: BPF
Default is LPF.

\end{itemize}

\sphinxlineitem{Returns}
\sphinxAtStartPar
\sphinxstylestrong{fft\_filter} \textendash{} Requested 2D Fourier filter.

\sphinxlineitem{Return type}
\sphinxAtStartPar
array

\end{description}\end{quote}

\end{fulllineitems}

\index{get\_lxly() (in module flatsky)@\spxentry{get\_lxly()}\spxextra{in module flatsky}}

\begin{fulllineitems}
\phantomsection\label{\detokenize{flatsky:flatsky.get_lxly}}
\pysigstartsignatures
\pysiglinewithargsret{\sphinxcode{\sphinxupquote{flatsky.}}\sphinxbfcode{\sphinxupquote{get\_lxly}}}{\sphinxparam{\DUrole{n}{flatskymapparams}}}{}
\pysigstopsignatures
\sphinxAtStartPar
return lx, ly modes (kx, ky Fourier modes) for a flatsky map grid.
\begin{quote}\begin{description}
\sphinxlineitem{Parameters}
\sphinxAtStartPar
\sphinxstyleliteralstrong{\sphinxupquote{flatskymyapparams}} (\sphinxstyleliteralemphasis{\sphinxupquote{list}}) \textendash{} {[}nx, ny, dx, dy{]} where ny, nx = flatskymap.shape; and dy, dx are the pixel resolution in arcminutes.
for example: {[}100, 100, 0.5, 0.5{]} is a 50’ x 50’ flatskymap that has dimensions 100 x 100 with dx = dy = 0.5 arcminutes.

\sphinxlineitem{Returns}
\sphinxAtStartPar
\sphinxstylestrong{lx, ly}

\sphinxlineitem{Return type}
\sphinxAtStartPar
array, shape is (ny, nx).

\end{description}\end{quote}

\end{fulllineitems}

\index{get\_lxly\_az\_angle() (in module flatsky)@\spxentry{get\_lxly\_az\_angle()}\spxextra{in module flatsky}}

\begin{fulllineitems}
\phantomsection\label{\detokenize{flatsky:flatsky.get_lxly_az_angle}}
\pysigstartsignatures
\pysiglinewithargsret{\sphinxcode{\sphinxupquote{flatsky.}}\sphinxbfcode{\sphinxupquote{get\_lxly\_az\_angle}}}{\sphinxparam{\DUrole{n}{lx}}\sphinxparamcomma \sphinxparam{\DUrole{n}{ly}}}{}
\pysigstopsignatures
\sphinxAtStartPar
azimuthal angle from lx, ly
\begin{quote}\begin{description}
\sphinxlineitem{Parameters}\begin{itemize}
\item {} 
\sphinxAtStartPar
\sphinxstyleliteralstrong{\sphinxupquote{lx}} (\sphinxstyleliteralemphasis{\sphinxupquote{array}}) \textendash{} lx modes

\item {} 
\sphinxAtStartPar
\sphinxstyleliteralstrong{\sphinxupquote{ly}} (\sphinxstyleliteralemphasis{\sphinxupquote{array}}) \textendash{} ly modes

\end{itemize}

\sphinxlineitem{Returns}
\sphinxAtStartPar
\sphinxstylestrong{phi} \textendash{} azimuthal angle

\sphinxlineitem{Return type}
\sphinxAtStartPar
array

\end{description}\end{quote}

\end{fulllineitems}

\index{make\_gaussian\_realisation() (in module flatsky)@\spxentry{make\_gaussian\_realisation()}\spxextra{in module flatsky}}

\begin{fulllineitems}
\phantomsection\label{\detokenize{flatsky:flatsky.make_gaussian_realisation}}
\pysigstartsignatures
\pysiglinewithargsret{\sphinxcode{\sphinxupquote{flatsky.}}\sphinxbfcode{\sphinxupquote{make\_gaussian\_realisation}}}{\sphinxparam{\DUrole{n}{mapparams}}\sphinxparamcomma \sphinxparam{\DUrole{n}{el}}\sphinxparamcomma \sphinxparam{\DUrole{n}{cl}}\sphinxparamcomma \sphinxparam{\DUrole{n}{cl2}\DUrole{o}{=}\DUrole{default_value}{None}}\sphinxparamcomma \sphinxparam{\DUrole{n}{cl12}\DUrole{o}{=}\DUrole{default_value}{None}}\sphinxparamcomma \sphinxparam{\DUrole{n}{cltwod}\DUrole{o}{=}\DUrole{default_value}{None}}\sphinxparamcomma \sphinxparam{\DUrole{n}{tf}\DUrole{o}{=}\DUrole{default_value}{None}}\sphinxparamcomma \sphinxparam{\DUrole{n}{bl}\DUrole{o}{=}\DUrole{default_value}{None}}\sphinxparamcomma \sphinxparam{\DUrole{n}{qu\_or\_eb}\DUrole{o}{=}\DUrole{default_value}{\textquotesingle{}qu\textquotesingle{}}}}{}
\pysigstopsignatures
\sphinxAtStartPar
Make gaussian realisation of flat sky map or 2maps based on the flatskymap parameters and the input power spectra.
Look into cl2map for a simple version.
\begin{quote}\begin{description}
\sphinxlineitem{Parameters}\begin{itemize}
\item {} 
\sphinxAtStartPar
\sphinxstyleliteralstrong{\sphinxupquote{flatskymyapparams}} (\sphinxstyleliteralemphasis{\sphinxupquote{list}}) \textendash{} {[}nx, ny, dx, dy{]} where ny, nx = flatskymap.shape; and dy, dx are the pixel resolution in arcminutes.
for example: {[}100, 100, 0.5, 0.5{]} is a 50’ x 50’ flatskymap that has dimensions 100 x 100 with dx = dy = 0.5 arcminutes.

\item {} 
\sphinxAtStartPar
\sphinxstyleliteralstrong{\sphinxupquote{el}} (\sphinxstyleliteralemphasis{\sphinxupquote{array}}) \textendash{} Multipoles over which the power spectrum is defined.

\item {} 
\sphinxAtStartPar
\sphinxstyleliteralstrong{\sphinxupquote{cl}} (\sphinxstyleliteralemphasis{\sphinxupquote{array}}) \textendash{} 1d vector of Cl auto\sphinxhyphen{}power spectra for map1.

\item {} 
\sphinxAtStartPar
\sphinxstyleliteralstrong{\sphinxupquote{cl2}} (\sphinxstyleliteralemphasis{\sphinxupquote{array}}\sphinxstyleliteralemphasis{\sphinxupquote{ (}}\sphinxstyleliteralemphasis{\sphinxupquote{optional}}\sphinxstyleliteralemphasis{\sphinxupquote{)}}) \textendash{} 1d vector of Cl2 auto\sphinxhyphen{}power spectra for map2.
Default is None. Used to generate correlated maps.

\item {} 
\sphinxAtStartPar
\sphinxstyleliteralstrong{\sphinxupquote{cl12}} (\sphinxstyleliteralemphasis{\sphinxupquote{array}}\sphinxstyleliteralemphasis{\sphinxupquote{ (}}\sphinxstyleliteralemphasis{\sphinxupquote{optional}}\sphinxstyleliteralemphasis{\sphinxupquote{)}}) \textendash{} 1d vector of Cl2 cross\sphinxhyphen{}power spectra of map1 and map2.
Default is None. Used to generate correlated maps.

\item {} 
\sphinxAtStartPar
\sphinxstyleliteralstrong{\sphinxupquote{cltwod}} (\sphinxstyleliteralemphasis{\sphinxupquote{array}}) \textendash{} 2D version of cl.
Default is None. Computed using 1d vector assuming azimuthal symmetry.

\item {} 
\sphinxAtStartPar
\sphinxstyleliteralstrong{\sphinxupquote{tf}} (\sphinxstyleliteralemphasis{\sphinxupquote{array}}) \textendash{} 2D filtering.
Default is None. Used to removed filtered modes.

\item {} 
\sphinxAtStartPar
\sphinxstyleliteralstrong{\sphinxupquote{bl}} (\sphinxstyleliteralemphasis{\sphinxupquote{array}}) \textendash{} 1d beam window function.
Default is None. Used for smoothing the maps.

\item {} 
\sphinxAtStartPar
\sphinxstyleliteralstrong{\sphinxupquote{qu\_or\_eb}} (\sphinxstyleliteralemphasis{\sphinxupquote{array}}) \textendash{} Generates TQU or TEB maps if cl, cl2, cl12 are supplied.
Default is ‘QU’.

\end{itemize}

\sphinxlineitem{Returns}
\sphinxAtStartPar
\sphinxstylestrong{sim\_map\_arr} \textendash{} sim\_map1: T\sphinxhyphen{}map.
if cl2 and cl12 are provided:
sim\_map2: Q or E map.
sim\_map3: U or B map.

\sphinxlineitem{Return type}
\sphinxAtStartPar
array.

\end{description}\end{quote}


\begin{sphinxseealso}{See also:}

\sphinxAtStartPar
{\hyperref[\detokenize{flatsky:flatsky.cl2map}]{\sphinxcrossref{\sphinxcode{\sphinxupquote{cl2map}}}}}


\end{sphinxseealso}


\end{fulllineitems}

\index{map2cl() (in module flatsky)@\spxentry{map2cl()}\spxextra{in module flatsky}}

\begin{fulllineitems}
\phantomsection\label{\detokenize{flatsky:flatsky.map2cl}}
\pysigstartsignatures
\pysiglinewithargsret{\sphinxcode{\sphinxupquote{flatsky.}}\sphinxbfcode{\sphinxupquote{map2cl}}}{\sphinxparam{\DUrole{n}{flatskymapparams}}\sphinxparamcomma \sphinxparam{\DUrole{n}{flatskymap1}}\sphinxparamcomma \sphinxparam{\DUrole{n}{flatskymap2}\DUrole{o}{=}\DUrole{default_value}{None}}\sphinxparamcomma \sphinxparam{\DUrole{n}{binsize}\DUrole{o}{=}\DUrole{default_value}{None}}}{}
\pysigstopsignatures
\sphinxAtStartPar
map2cl module \sphinxhyphen{} get the auto\sphinxhyphen{}/cross\sphinxhyphen{}power spectra of map/maps
\begin{quote}\begin{description}
\sphinxlineitem{Parameters}\begin{itemize}
\item {} 
\sphinxAtStartPar
\sphinxstyleliteralstrong{\sphinxupquote{flatskymyapparams}} (\sphinxstyleliteralemphasis{\sphinxupquote{list}}) \textendash{} {[}nx, ny, dx, dy{]} where ny, nx = flatskymap.shape; and dy, dx are the pixel resolution in arcminutes.
for example: {[}100, 100, 0.5, 0.5{]} is a 50’ x 50’ flatskymap that has dimensions 100 x 100 with dx = dy = 0.5 arcminutes.

\item {} 
\sphinxAtStartPar
\sphinxstyleliteralstrong{\sphinxupquote{flatskymap1}} (\sphinxstyleliteralemphasis{\sphinxupquote{array}}) \textendash{} flatskymap1 with dimensions (ny, nx).

\item {} 
\sphinxAtStartPar
\sphinxstyleliteralstrong{\sphinxupquote{flatskymap2}} (\sphinxstyleliteralemphasis{\sphinxupquote{array}}\sphinxstyleliteralemphasis{\sphinxupquote{ (}}\sphinxstyleliteralemphasis{\sphinxupquote{Optional}}\sphinxstyleliteralemphasis{\sphinxupquote{)}}) \textendash{} flatskymap2 with dimensions (ny, nx).
Default is None.
If None, compute the auto\sphinxhyphen{}spectrum of flatskymap1.
If not None, compute the cross\sphinxhyphen{}spectrum between flatskymap1 and flatskymap2.

\item {} 
\sphinxAtStartPar
\sphinxstyleliteralstrong{\sphinxupquote{binsize}} (\sphinxstyleliteralemphasis{\sphinxupquote{int}}) \textendash{} el bins. computed automatically based on the fft grid spacing if None.

\end{itemize}

\sphinxlineitem{Returns}
\sphinxAtStartPar
\begin{itemize}
\item {} 
\sphinxAtStartPar
\sphinxstylestrong{el} (\sphinxstyleemphasis{array}) \textendash{} Multipoles over which the power spectrum is defined.

\item {} 
\sphinxAtStartPar
\sphinxstylestrong{cl} (\sphinxstyleemphasis{array}) \textendash{} auto/cross power spectrum.

\end{itemize}


\end{description}\end{quote}

\end{fulllineitems}

\index{radial\_profile() (in module flatsky)@\spxentry{radial\_profile()}\spxextra{in module flatsky}}

\begin{fulllineitems}
\phantomsection\label{\detokenize{flatsky:flatsky.radial_profile}}
\pysigstartsignatures
\pysiglinewithargsret{\sphinxcode{\sphinxupquote{flatsky.}}\sphinxbfcode{\sphinxupquote{radial\_profile}}}{\sphinxparam{\DUrole{n}{z}}\sphinxparamcomma \sphinxparam{\DUrole{n}{xy}\DUrole{o}{=}\DUrole{default_value}{None}}\sphinxparamcomma \sphinxparam{\DUrole{n}{bin\_size}\DUrole{o}{=}\DUrole{default_value}{1.0}}\sphinxparamcomma \sphinxparam{\DUrole{n}{minbin}\DUrole{o}{=}\DUrole{default_value}{0.0}}\sphinxparamcomma \sphinxparam{\DUrole{n}{maxbin}\DUrole{o}{=}\DUrole{default_value}{10.0}}\sphinxparamcomma \sphinxparam{\DUrole{n}{to\_arcmins}\DUrole{o}{=}\DUrole{default_value}{1}}\sphinxparamcomma \sphinxparam{\DUrole{n}{get\_errors}\DUrole{o}{=}\DUrole{default_value}{1}}}{}
\pysigstopsignatures
\sphinxAtStartPar
get the radial profile of an image (both real and fourier space).
Can be used to compute radial profile of stacked profiles or 2D power spectrum.
\begin{quote}\begin{description}
\sphinxlineitem{Parameters}\begin{itemize}
\item {} 
\sphinxAtStartPar
\sphinxstyleliteralstrong{\sphinxupquote{z}} (\sphinxstyleliteralemphasis{\sphinxupquote{array}}) \textendash{} image to get the radial profile.

\item {} 
\sphinxAtStartPar
\sphinxstyleliteralstrong{\sphinxupquote{xy}} (\sphinxstyleliteralemphasis{\sphinxupquote{array}}) \textendash{} x and y grid. Same shape as the image z.
Default is None.
If None,
x, y = np.indices(image.shape)

\item {} 
\sphinxAtStartPar
\sphinxstyleliteralstrong{\sphinxupquote{bin\_size}} (\sphinxstyleliteralemphasis{\sphinxupquote{float}}) \textendash{} radial binning factor.
default is 1.

\item {} 
\sphinxAtStartPar
\sphinxstyleliteralstrong{\sphinxupquote{minbin}} (\sphinxstyleliteralemphasis{\sphinxupquote{float}}) \textendash{} minimum bin for radial profile
default is 0.

\item {} 
\sphinxAtStartPar
\sphinxstyleliteralstrong{\sphinxupquote{maxbin}} (\sphinxstyleliteralemphasis{\sphinxupquote{float}}) \textendash{} minimum bin for radial profile
default is 10.

\item {} 
\sphinxAtStartPar
\sphinxstyleliteralstrong{\sphinxupquote{to\_arcmins}} (\sphinxstyleliteralemphasis{\sphinxupquote{bool}}) \textendash{} If set, then xy are assumed to be in degrees and multipled by 60 to convert to arcmins.

\item {} 
\sphinxAtStartPar
\sphinxstyleliteralstrong{\sphinxupquote{get\_errors}} (\sphinxstyleliteralemphasis{\sphinxupquote{bool}}) \textendash{} obtain scatter in each bin.
This is not the error due to variance. Just the sample variance.
Default is True.

\end{itemize}

\sphinxlineitem{Returns}
\sphinxAtStartPar
\sphinxstylestrong{radprf} \textendash{} Array with three elements cotaining
radprf{[}:,0{]} = radial bins
radprf{[}:,1{]} = radial binned values
if get\_errors:
radprf{[}:,2{]} = radial bin errors.

\sphinxlineitem{Return type}
\sphinxAtStartPar
array.

\end{description}\end{quote}

\end{fulllineitems}

\index{wiener\_filter() (in module flatsky)@\spxentry{wiener\_filter()}\spxextra{in module flatsky}}

\begin{fulllineitems}
\phantomsection\label{\detokenize{flatsky:flatsky.wiener_filter}}
\pysigstartsignatures
\pysiglinewithargsret{\sphinxcode{\sphinxupquote{flatsky.}}\sphinxbfcode{\sphinxupquote{wiener\_filter}}}{\sphinxparam{\DUrole{n}{flatskymyapparams}}\sphinxparamcomma \sphinxparam{\DUrole{n}{cl\_signal}}\sphinxparamcomma \sphinxparam{\DUrole{n}{cl\_noise}}\sphinxparamcomma \sphinxparam{\DUrole{n}{el}\DUrole{o}{=}\DUrole{default_value}{None}}}{}
\pysigstopsignatures
\sphinxAtStartPar
Get 2D Wiener filter.
\begin{equation*}
\begin{split}W(\ell) = \frac{ C_{\ell}^{\rm signal} } {C_{\ell}^{\rm signal} + C_{\ell}^{\rm noise}}\end{split}
\end{equation*}\begin{quote}\begin{description}
\sphinxlineitem{Parameters}\begin{itemize}
\item {} 
\sphinxAtStartPar
\sphinxstyleliteralstrong{\sphinxupquote{flatskymyapparams}} (\sphinxstyleliteralemphasis{\sphinxupquote{list}}) \textendash{} {[}nx, ny, dx, dy{]} where ny, nx = flatskymap.shape; and dy, dx are the pixel resolution in arcminutes.
for example: {[}100, 100, 0.5, 0.5{]} is a 50’ x 50’ flatskymap that has dimensions 100 x 100 with dx = dy = 0.5 arcminutes.

\item {} 
\sphinxAtStartPar
\sphinxstyleliteralstrong{\sphinxupquote{cl\_signal}} (\sphinxstyleliteralemphasis{\sphinxupquote{array}}) \textendash{} Power spectrum of the signal component.

\item {} 
\sphinxAtStartPar
\sphinxstyleliteralstrong{\sphinxupquote{cl\_noise}} (\sphinxstyleliteralemphasis{\sphinxupquote{array}}) \textendash{} Power spectrum of the noise component.

\item {} 
\sphinxAtStartPar
\sphinxstyleliteralstrong{\sphinxupquote{el}} (\sphinxstyleliteralemphasis{\sphinxupquote{array}}\sphinxstyleliteralemphasis{\sphinxupquote{ (}}\sphinxstyleliteralemphasis{\sphinxupquote{optional}}\sphinxstyleliteralemphasis{\sphinxupquote{)}}) \textendash{} Multipole over which the signal / noise spectra are defined.
Default is None and el will be np.arange( len(cl\_signal) )

\end{itemize}

\sphinxlineitem{Returns}
\sphinxAtStartPar
\sphinxstylestrong{wiener\_filter} \textendash{} 2D Wiener filter.

\sphinxlineitem{Return type}
\sphinxAtStartPar
array

\end{description}\end{quote}

\end{fulllineitems}


\sphinxstepscope


\chapter{Inpaint module}
\label{\detokenize{inpaint:module-inpaint}}\label{\detokenize{inpaint:inpaint-module}}\label{\detokenize{inpaint::doc}}\index{module@\spxentry{module}!inpaint@\spxentry{inpaint}}\index{inpaint@\spxentry{inpaint}!module@\spxentry{module}}\index{calccov() (in module inpaint)@\spxentry{calccov()}\spxextra{in module inpaint}}

\begin{fulllineitems}
\phantomsection\label{\detokenize{inpaint:inpaint.calccov}}
\pysigstartsignatures
\pysiglinewithargsret{\sphinxcode{\sphinxupquote{inpaint.}}\sphinxbfcode{\sphinxupquote{calccov}}}{\sphinxparam{\DUrole{n}{sim\_mat}}\sphinxparamcomma \sphinxparam{\DUrole{n}{noofsims}}\sphinxparamcomma \sphinxparam{\DUrole{n}{npixels}}}{}
\pysigstopsignatures
\sphinxAtStartPar
Computer pixel\sphinxhyphen{}pixel covarinace based on simulations.
\begin{quote}\begin{description}
\sphinxlineitem{Parameters}\begin{itemize}
\item {} 
\sphinxAtStartPar
\sphinxstyleliteralstrong{\sphinxupquote{sim\_mat}} (\sphinxstyleliteralemphasis{\sphinxupquote{array}}) \textendash{} nd array multiple sim maps used for covariance.

\item {} 
\sphinxAtStartPar
\sphinxstyleliteralstrong{\sphinxupquote{noofsims}} (\sphinxstyleliteralemphasis{\sphinxupquote{int}}) \textendash{} number of simulations.

\item {} 
\sphinxAtStartPar
\sphinxstyleliteralstrong{\sphinxupquote{npixels}} (\sphinxstyleliteralemphasis{\sphinxupquote{int}}) \textendash{} Map dimension (number of pixels).

\end{itemize}

\sphinxlineitem{Returns}
\sphinxAtStartPar
\sphinxstylestrong{cov} \textendash{} covarinace array.
Same as np.cov(sim\_mat).

\sphinxlineitem{Return type}
\sphinxAtStartPar
array

\end{description}\end{quote}

\end{fulllineitems}

\index{get\_covariance() (in module inpaint)@\spxentry{get\_covariance()}\spxextra{in module inpaint}}

\begin{fulllineitems}
\phantomsection\label{\detokenize{inpaint:inpaint.get_covariance}}
\pysigstartsignatures
\pysiglinewithargsret{\sphinxcode{\sphinxupquote{inpaint.}}\sphinxbfcode{\sphinxupquote{get\_covariance}}}{\sphinxparam{\DUrole{n}{ra\_grid}}\sphinxparamcomma \sphinxparam{\DUrole{n}{dec\_grid}}\sphinxparamcomma \sphinxparam{\DUrole{n}{mapparams}}\sphinxparamcomma \sphinxparam{\DUrole{n}{el}}\sphinxparamcomma \sphinxparam{\DUrole{n}{cl\_dic}}\sphinxparamcomma \sphinxparam{\DUrole{n}{bl}}\sphinxparamcomma \sphinxparam{\DUrole{n}{nl\_dic}}\sphinxparamcomma \sphinxparam{\DUrole{n}{noofsims}}\sphinxparamcomma \sphinxparam{\DUrole{n}{mask\_radius\_inner}}\sphinxparamcomma \sphinxparam{\DUrole{n}{mask\_radius\_outer}}\sphinxparamcomma \sphinxparam{\DUrole{n}{low\_pass\_cutoff}\DUrole{o}{=}\DUrole{default_value}{1}}\sphinxparamcomma \sphinxparam{\DUrole{n}{maxel\_for\_grad\_filter}\DUrole{o}{=}\DUrole{default_value}{None}}}{}
\pysigstopsignatures
\sphinxAtStartPar
Compute the covariance between regions R1 and R2 required for inpainting.
\begin{equation*}
\begin{split}\hat{T}_{1} = \tilde{T}_{1} + {\bf \hat{C}}_{12}  {\bf \hat{C}}_{22}^{-1} (T_{2} - \tilde{T}_{2})\end{split}
\end{equation*}\begin{quote}\begin{description}
\sphinxlineitem{Parameters}\begin{itemize}
\item {} 
\sphinxAtStartPar
\sphinxstyleliteralstrong{\sphinxupquote{ra\_grid}} (\sphinxstyleliteralemphasis{\sphinxupquote{array}}) \textendash{} ra\_grid in degrees or arcmins for the flatsky map.

\item {} 
\sphinxAtStartPar
\sphinxstyleliteralstrong{\sphinxupquote{dec\_grid}} (\sphinxstyleliteralemphasis{\sphinxupquote{array}}) \textendash{} dec\_grid in degress or arcmins for the flatsky map.

\item {} 
\sphinxAtStartPar
\sphinxstyleliteralstrong{\sphinxupquote{mapparams}} (\sphinxstyleliteralemphasis{\sphinxupquote{list}}) \textendash{} {[}nx, ny, dx, dy{]} where ny, nx = flatskymap.shape; and dy, dx are the pixel resolution in arcminutes.
for example: {[}100, 100, 0.5, 0.5{]} is a 50’ x 50’ flatskymap that has dimensions 100 x 100 with dx = dy = 0.5 arcminutes.

\item {} 
\sphinxAtStartPar
\sphinxstyleliteralstrong{\sphinxupquote{el}} (\sphinxstyleliteralemphasis{\sphinxupquote{array}}) \textendash{} Multipoles over which the power spectrum is defined.

\item {} 
\sphinxAtStartPar
\sphinxstyleliteralstrong{\sphinxupquote{cl\_dic}} (\sphinxstyleliteralemphasis{\sphinxupquote{dict}}) \textendash{} Signal power spectra dictionary.
Keys are “TT” for T\sphinxhyphen{}only inpainting;
“TT”, “EE”, “TE” for T/Q/U inpainting.

\item {} 
\sphinxAtStartPar
\sphinxstyleliteralstrong{\sphinxupquote{bl}} (\sphinxstyleliteralemphasis{\sphinxupquote{array}}) \textendash{} 1d beam window function.
Default is None. Used for smoothing the maps.

\item {} 
\sphinxAtStartPar
\sphinxstyleliteralstrong{\sphinxupquote{nl\_dic}} (\sphinxstyleliteralemphasis{\sphinxupquote{dict}}) \textendash{} Noise power spectra dictionary.
Keys are “TT” for T\sphinxhyphen{}only inpainting;
“TT”, “EE”, “TE” for T/Q/U inpainting.

\item {} 
\sphinxAtStartPar
\sphinxstyleliteralstrong{\sphinxupquote{noofsims}} (\sphinxstyleliteralemphasis{\sphinxupquote{int}}) \textendash{} number of simulations used for covariance calculation.

\item {} 
\sphinxAtStartPar
\sphinxstyleliteralstrong{\sphinxupquote{mask\_radius\_inner}} (\sphinxstyleliteralemphasis{\sphinxupquote{float}}) \textendash{} Inner radius of region R1 to inpaint in arcmins.

\item {} 
\sphinxAtStartPar
\sphinxstyleliteralstrong{\sphinxupquote{mask\_radius\_outer}} (\sphinxstyleliteralemphasis{\sphinxupquote{float}}) \textendash{} Outer radius of region R2 to inpaint in arcmins.

\item {} 
\sphinxAtStartPar
\sphinxstyleliteralstrong{\sphinxupquote{low\_pass\_cutoff}} (\sphinxstyleliteralemphasis{\sphinxupquote{bool}}) \textendash{} Low pass filter the maps before inpainting.
Default is True.

\item {} 
\sphinxAtStartPar
\sphinxstyleliteralstrong{\sphinxupquote{maxel\_for\_grad\_filter}} (\sphinxstyleliteralemphasis{\sphinxupquote{int}}) \textendash{} lmax for the LPF above.
Default is None in which it will be calculated based on the radius of the inner region.

\end{itemize}

\sphinxlineitem{Returns}
\sphinxAtStartPar
\sphinxstylestrong{sigma\_dic} \textendash{} Covariance dictionary containing the covariance between R12, R22, and R22\_inv.
See the equation above.

\sphinxlineitem{Return type}
\sphinxAtStartPar
dict (Optional)

\end{description}\end{quote}


\begin{sphinxseealso}{See also:}

\sphinxAtStartPar
{\hyperref[\detokenize{inpaint:inpaint.inpainting}]{\sphinxcrossref{\sphinxcode{\sphinxupquote{inpainting}}}}}


\end{sphinxseealso}


\end{fulllineitems}

\index{get\_mask\_indices() (in module inpaint)@\spxentry{get\_mask\_indices()}\spxextra{in module inpaint}}

\begin{fulllineitems}
\phantomsection\label{\detokenize{inpaint:inpaint.get_mask_indices}}
\pysigstartsignatures
\pysiglinewithargsret{\sphinxcode{\sphinxupquote{inpaint.}}\sphinxbfcode{\sphinxupquote{get\_mask\_indices}}}{\sphinxparam{\DUrole{n}{ra\_grid}}\sphinxparamcomma \sphinxparam{\DUrole{n}{dec\_grid}}\sphinxparamcomma \sphinxparam{\DUrole{n}{mask\_radius\_inner}}\sphinxparamcomma \sphinxparam{\DUrole{n}{mask\_radius\_outer}}\sphinxparamcomma \sphinxparam{\DUrole{n}{square}\DUrole{o}{=}\DUrole{default_value}{0}}\sphinxparamcomma \sphinxparam{\DUrole{n}{in\_arcmins}\DUrole{o}{=}\DUrole{default_value}{1}}}{}
\pysigstopsignatures
\sphinxAtStartPar
Get the pixel indices in regions R1 and R2 for inpainting.
\begin{quote}\begin{description}
\sphinxlineitem{Parameters}\begin{itemize}
\item {} 
\sphinxAtStartPar
\sphinxstyleliteralstrong{\sphinxupquote{ra\_grid}} (\sphinxstyleliteralemphasis{\sphinxupquote{array}}) \textendash{} ra\_grid in degrees or arcmins for the flatsky map.

\item {} 
\sphinxAtStartPar
\sphinxstyleliteralstrong{\sphinxupquote{dec\_grid}} (\sphinxstyleliteralemphasis{\sphinxupquote{array}}) \textendash{} dec\_grid in degress or arcmins for the flatsky map.

\item {} 
\sphinxAtStartPar
\sphinxstyleliteralstrong{\sphinxupquote{mask\_radius\_inner}} (\sphinxstyleliteralemphasis{\sphinxupquote{float}}) \textendash{} Inner radius of region R1 to inpaint in arcmins.

\item {} 
\sphinxAtStartPar
\sphinxstyleliteralstrong{\sphinxupquote{mask\_radius\_outer}} (\sphinxstyleliteralemphasis{\sphinxupquote{float}}) \textendash{} Outer radius of region R2 to inpaint in arcmins.

\item {} 
\sphinxAtStartPar
\sphinxstyleliteralstrong{\sphinxupquote{low\_pass\_cutoff}} (\sphinxstyleliteralemphasis{\sphinxupquote{bool}}) \textendash{} Low pass filter the maps before inpainting.
Default is True.

\item {} 
\sphinxAtStartPar
\sphinxstyleliteralstrong{\sphinxupquote{square}} (\sphinxstyleliteralemphasis{\sphinxupquote{bool}}) \textendash{} if True, returns a square.
else, returns a cirucalr region.
Default is Circle.

\item {} 
\sphinxAtStartPar
\sphinxstyleliteralstrong{\sphinxupquote{in\_arcmins}} (\sphinxstyleliteralemphasis{\sphinxupquote{bool}}) \textendash{} Supplied grid are in arcmins.
default is True.
If False, degrees is assumed and will be converted to arcmins.

\end{itemize}

\sphinxlineitem{Returns}
\sphinxAtStartPar
\begin{itemize}
\item {} 
\sphinxAtStartPar
\sphinxstylestrong{inds\_inner} (\sphinxstyleemphasis{array}) \textendash{} pixel indices of the inner region R1.

\item {} 
\sphinxAtStartPar
\sphinxstylestrong{inds\_outer} (\sphinxstyleemphasis{array}) \textendash{} pixel indices of the outer region R2.

\end{itemize}


\end{description}\end{quote}

\end{fulllineitems}

\index{inpainting() (in module inpaint)@\spxentry{inpainting()}\spxextra{in module inpaint}}

\begin{fulllineitems}
\phantomsection\label{\detokenize{inpaint:inpaint.inpainting}}
\pysigstartsignatures
\pysiglinewithargsret{\sphinxcode{\sphinxupquote{inpaint.}}\sphinxbfcode{\sphinxupquote{inpainting}}}{\sphinxparam{\DUrole{n}{map\_dic\_to\_inpaint}}\sphinxparamcomma \sphinxparam{\DUrole{n}{ra\_grid}}\sphinxparamcomma \sphinxparam{\DUrole{n}{dec\_grid}}\sphinxparamcomma \sphinxparam{\DUrole{n}{mapparams}}\sphinxparamcomma \sphinxparam{\DUrole{n}{el}}\sphinxparamcomma \sphinxparam{\DUrole{n}{cl\_dic}}\sphinxparamcomma \sphinxparam{\DUrole{n}{bl}}\sphinxparamcomma \sphinxparam{\DUrole{n}{nl\_dic}}\sphinxparamcomma \sphinxparam{\DUrole{n}{noofsims}}\sphinxparamcomma \sphinxparam{\DUrole{n}{mask\_radius\_inner}}\sphinxparamcomma \sphinxparam{\DUrole{n}{mask\_radius\_outer}}\sphinxparamcomma \sphinxparam{\DUrole{n}{low\_pass\_cutoff}\DUrole{o}{=}\DUrole{default_value}{1}}\sphinxparamcomma \sphinxparam{\DUrole{n}{maxel\_for\_grad\_filter}\DUrole{o}{=}\DUrole{default_value}{None}}\sphinxparamcomma \sphinxparam{\DUrole{n}{intrp\_r1\_before\_lpf}\DUrole{o}{=}\DUrole{default_value}{0}}\sphinxparamcomma \sphinxparam{\DUrole{n}{mask\_inner}\DUrole{o}{=}\DUrole{default_value}{0}}\sphinxparamcomma \sphinxparam{\DUrole{n}{sigma\_dic}\DUrole{o}{=}\DUrole{default_value}{None}}\sphinxparamcomma \sphinxparam{\DUrole{n}{use\_original}\DUrole{o}{=}\DUrole{default_value}{False}}\sphinxparamcomma \sphinxparam{\DUrole{n}{use\_cons\_gau\_sims}\DUrole{o}{=}\DUrole{default_value}{True}}}{}
\pysigstopsignatures
\sphinxAtStartPar
Perform inpainting. Can perform joint inpainting of T/Q/U maps using the cross\sphinxhyphen{}spectra between them.
\#\#\#mask\_inner = 1: The inner region is masked before the LPF. Might be useful in the presence of bright SZ signal at the centre.
\begin{equation*}
\begin{split}\hat{T}_{1} = \tilde{T}_{1} + {\bf \hat{C}}_{12}  {\bf \hat{C}}_{22}^{-1} (T_{2} - \tilde{T}_{2})\end{split}
\end{equation*}\begin{quote}\begin{description}
\sphinxlineitem{Parameters}\begin{itemize}
\item {} 
\sphinxAtStartPar
\sphinxstyleliteralstrong{\sphinxupquote{map\_dic\_to\_inpaint}} (\sphinxstyleliteralemphasis{\sphinxupquote{dict}}) \textendash{} flatsky map dict to inpaint.
Keys must be {[}“T”, “Q”, “U”{]}.

\item {} 
\sphinxAtStartPar
\sphinxstyleliteralstrong{\sphinxupquote{ra\_grid}} (\sphinxstyleliteralemphasis{\sphinxupquote{array}}) \textendash{} ra\_grid in degrees or arcmins for the flatsky map.

\item {} 
\sphinxAtStartPar
\sphinxstyleliteralstrong{\sphinxupquote{dec\_grid}} (\sphinxstyleliteralemphasis{\sphinxupquote{array}}) \textendash{} dec\_grid in degress or arcmins for the flatsky map.

\item {} 
\sphinxAtStartPar
\sphinxstyleliteralstrong{\sphinxupquote{mapparams}} (\sphinxstyleliteralemphasis{\sphinxupquote{list}}) \textendash{} {[}nx, ny, dx, dy{]} where ny, nx = flatskymap.shape; and dy, dx are the pixel resolution in arcminutes.
for example: {[}100, 100, 0.5, 0.5{]} is a 50’ x 50’ flatskymap that has dimensions 100 x 100 with dx = dy = 0.5 arcminutes.

\item {} 
\sphinxAtStartPar
\sphinxstyleliteralstrong{\sphinxupquote{el}} (\sphinxstyleliteralemphasis{\sphinxupquote{array}}) \textendash{} Multipoles over which the power spectrum is defined.

\item {} 
\sphinxAtStartPar
\sphinxstyleliteralstrong{\sphinxupquote{cl\_dic}} (\sphinxstyleliteralemphasis{\sphinxupquote{dict}}) \textendash{} Signal power spectra dictionary.
Keys are “TT” for T\sphinxhyphen{}only inpainting;
“TT”, “EE”, “TE” for T/Q/U inpainting.

\item {} 
\sphinxAtStartPar
\sphinxstyleliteralstrong{\sphinxupquote{bl}} (\sphinxstyleliteralemphasis{\sphinxupquote{array}}) \textendash{} 1d beam window function.
Default is None. Used for smoothing the maps.

\item {} 
\sphinxAtStartPar
\sphinxstyleliteralstrong{\sphinxupquote{nl\_dic}} (\sphinxstyleliteralemphasis{\sphinxupquote{dict}}) \textendash{} Noise power spectra dictionary.
Keys are “TT” for T\sphinxhyphen{}only inpainting;
“TT”, “EE”, “TE” for T/Q/U inpainting.

\item {} 
\sphinxAtStartPar
\sphinxstyleliteralstrong{\sphinxupquote{noofsims}} (\sphinxstyleliteralemphasis{\sphinxupquote{int}}) \textendash{} number of simulations used for covariance calculation.

\item {} 
\sphinxAtStartPar
\sphinxstyleliteralstrong{\sphinxupquote{mask\_radius\_inner}} (\sphinxstyleliteralemphasis{\sphinxupquote{float}}) \textendash{} Inner radius of region R1 to inpaint in arcmins.

\item {} 
\sphinxAtStartPar
\sphinxstyleliteralstrong{\sphinxupquote{mask\_radius\_outer}} (\sphinxstyleliteralemphasis{\sphinxupquote{float}}) \textendash{} Outer radius of region R2 to inpaint in arcmins.

\item {} 
\sphinxAtStartPar
\sphinxstyleliteralstrong{\sphinxupquote{low\_pass\_cutoff}} (\sphinxstyleliteralemphasis{\sphinxupquote{bool}}) \textendash{} Low pass filter the maps before inpainting.
Default is True.

\item {} 
\sphinxAtStartPar
\sphinxstyleliteralstrong{\sphinxupquote{maxel\_for\_grad\_filter}} (\sphinxstyleliteralemphasis{\sphinxupquote{int}}) \textendash{} lmax for the LPF above.
Default is None in which it will be calculated based on the radius of the inner region.

\item {} 
\sphinxAtStartPar
\sphinxstyleliteralstrong{\sphinxupquote{intrp\_r1\_before\_lpf}} (\sphinxstyleliteralemphasis{\sphinxupquote{bool}}) \textendash{} Interpolate R1 before inpainting.
Default is False.

\item {} 
\sphinxAtStartPar
\sphinxstyleliteralstrong{\sphinxupquote{mask\_inner}} (\sphinxstyleliteralemphasis{\sphinxupquote{bool}}) \textendash{} Mask inner region before inpainting.
Default is False.

\item {} 
\sphinxAtStartPar
\sphinxstyleliteralstrong{\sphinxupquote{sigma\_dic}} (\sphinxstyleliteralemphasis{\sphinxupquote{dict}}\sphinxstyleliteralemphasis{\sphinxupquote{ (}}\sphinxstyleliteralemphasis{\sphinxupquote{Optional}}\sphinxstyleliteralemphasis{\sphinxupquote{)}}) \textendash{} Covariance dictionary containing the covariance between R12, R22, and R22\_inv.
See the equation above.
If None, this will be calculated on the fly.

\item {} 
\sphinxAtStartPar
\sphinxstyleliteralstrong{\sphinxupquote{use\_original}} (\sphinxstyleliteralemphasis{\sphinxupquote{bool}}) \textendash{} Do not inpaint. Return the same map.
Default is False.

\item {} 
\sphinxAtStartPar
\sphinxstyleliteralstrong{\sphinxupquote{use\_cons\_gau\_sims}} (\sphinxstyleliteralemphasis{\sphinxupquote{bool}}) \textendash{} Use constrained Gaussian realisations or not.
If False, fields with tilde will be set to zero and there will be no randomness.
(i.e:) Just interpolate R1 based on R2 and the covariance.
default is True.

\end{itemize}

\sphinxlineitem{Returns}
\sphinxAtStartPar
\begin{itemize}
\item {} 
\sphinxAtStartPar
\sphinxstylestrong{cmb\_inpainted\_map} (\sphinxstyleemphasis{array}) \textendash{} inpainted CMB region. All pixels other than R1 will be zero in this map.

\item {} 
\sphinxAtStartPar
\sphinxstylestrong{inpainted\_map} (\sphinxstyleemphasis{array}) \textendash{} inpainted map.

\item {} 
\sphinxAtStartPar
\sphinxstylestrong{map\_to\_inpaint} (\sphinxstyleemphasis{array}) \textendash{} original map used for inpainting.

\end{itemize}


\end{description}\end{quote}


\begin{sphinxseealso}{See also:}

\sphinxAtStartPar
{\hyperref[\detokenize{inpaint:inpaint.get_covariance}]{\sphinxcrossref{\sphinxcode{\sphinxupquote{get\_covariance}}}}}


\end{sphinxseealso}


\end{fulllineitems}

\index{masking\_for\_filtering() (in module inpaint)@\spxentry{masking\_for\_filtering()}\spxextra{in module inpaint}}

\begin{fulllineitems}
\phantomsection\label{\detokenize{inpaint:inpaint.masking_for_filtering}}
\pysigstartsignatures
\pysiglinewithargsret{\sphinxcode{\sphinxupquote{inpaint.}}\sphinxbfcode{\sphinxupquote{masking\_for\_filtering}}}{\sphinxparam{\DUrole{n}{ra\_grid}}\sphinxparamcomma \sphinxparam{\DUrole{n}{dec\_grid}}\sphinxparamcomma \sphinxparam{\DUrole{n}{mask\_radius}\DUrole{o}{=}\DUrole{default_value}{2.0}}\sphinxparamcomma \sphinxparam{\DUrole{n}{taper\_radius}\DUrole{o}{=}\DUrole{default_value}{6.0}}\sphinxparamcomma \sphinxparam{\DUrole{n}{apodise}\DUrole{o}{=}\DUrole{default_value}{True}}\sphinxparamcomma \sphinxparam{\DUrole{n}{in\_arcmins}\DUrole{o}{=}\DUrole{default_value}{True}}}{}
\pysigstopsignatures
\sphinxAtStartPar
Mask regions before filtering. Returns an apodised (or binary) mask.
Default values correspond to ACT/SPT/SO/CMB\sphinxhyphen{}S4\sphinxhyphen{}like beams.
\begin{quote}\begin{description}
\sphinxlineitem{Parameters}\begin{itemize}
\item {} 
\sphinxAtStartPar
\sphinxstyleliteralstrong{\sphinxupquote{ra\_grid}} (\sphinxstyleliteralemphasis{\sphinxupquote{array}}) \textendash{} ra\_grid in degrees or arcmins for the flatsky map.

\item {} 
\sphinxAtStartPar
\sphinxstyleliteralstrong{\sphinxupquote{dec\_grid}} (\sphinxstyleliteralemphasis{\sphinxupquote{array}}) \textendash{} dec\_grid in degress or arcmins for the flatsky map.

\item {} 
\sphinxAtStartPar
\sphinxstyleliteralstrong{\sphinxupquote{mask\_radius}} (\sphinxstyleliteralemphasis{\sphinxupquote{float}}) \textendash{} radius for the masked region in arcmins.
default is 2 arcmins.

\item {} 
\sphinxAtStartPar
\sphinxstyleliteralstrong{\sphinxupquote{taper\_radius}} (\sphinxstyleliteralemphasis{\sphinxupquote{float}}) \textendash{} raptering radius in arcmins for the mask.
default is 6 arcmins.

\item {} 
\sphinxAtStartPar
\sphinxstyleliteralstrong{\sphinxupquote{apodise}} (\sphinxstyleliteralemphasis{\sphinxupquote{bool}}) \textendash{} If True, apodises the mask. Otherwise, return a binary mask.
Default is True.

\item {} 
\sphinxAtStartPar
\sphinxstyleliteralstrong{\sphinxupquote{in\_arcmins}} (\sphinxstyleliteralemphasis{\sphinxupquote{bool}}) \textendash{} Supplied grid are in arcmins.
default is True.
If False, degrees is assumed and will be converted to arcmins.

\end{itemize}

\sphinxlineitem{Returns}
\sphinxAtStartPar
\sphinxstylestrong{mask} \textendash{} mask corresponding to the parameter.

\sphinxlineitem{Return type}
\sphinxAtStartPar
array, shape is ra\_grid.shape.

\end{description}\end{quote}

\end{fulllineitems}


\sphinxstepscope


\chapter{Tools module}
\label{\detokenize{tools:module-tools}}\label{\detokenize{tools:tools-module}}\label{\detokenize{tools::doc}}\index{module@\spxentry{module}!tools@\spxentry{tools}}\index{tools@\spxentry{tools}!module@\spxentry{module}}\index{get\_blsqinv() (in module tools)@\spxentry{get\_blsqinv()}\spxextra{in module tools}}

\begin{fulllineitems}
\phantomsection\label{\detokenize{tools:tools.get_blsqinv}}
\pysigstartsignatures
\pysiglinewithargsret{\sphinxcode{\sphinxupquote{tools.}}\sphinxbfcode{\sphinxupquote{get\_blsqinv}}}{\sphinxparam{\DUrole{n}{beamval}}\sphinxparamcomma \sphinxparam{\DUrole{n}{el}}\sphinxparamcomma \sphinxparam{\DUrole{n}{make\_2d}\DUrole{o}{=}\DUrole{default_value}{0}}\sphinxparamcomma \sphinxparam{\DUrole{n}{mapparams}\DUrole{o}{=}\DUrole{default_value}{None}}}{}
\pysigstopsignatures
\sphinxAtStartPar
Get the inverse of the beam window function sqaured.
\begin{quote}\begin{description}
\sphinxlineitem{Parameters}\begin{itemize}
\item {} 
\sphinxAtStartPar
\sphinxstyleliteralstrong{\sphinxupquote{beamval}} (\sphinxstyleliteralemphasis{\sphinxupquote{float}}) \textendash{} Beam FWHM in arcmins.

\item {} 
\sphinxAtStartPar
\sphinxstyleliteralstrong{\sphinxupquote{el}} (\sphinxstyleliteralemphasis{\sphinxupquote{array}}) \textendash{} Multipoles over which the window function must be defined.

\item {} 
\sphinxAtStartPar
\sphinxstyleliteralstrong{\sphinxupquote{make\_2d}} (\sphinxstyleliteralemphasis{\sphinxupquote{bool}}) \textendash{} Convert to 2D if desired.
Default is False.

\item {} 
\sphinxAtStartPar
\sphinxstyleliteralstrong{\sphinxupquote{mapparams}} (\sphinxstyleliteralemphasis{\sphinxupquote{list}}) \textendash{} {[}nx, ny, dx, dy{]} where ny, nx = flatskymap.shape; and dy, dx are the pixel resolution in arcminutes.
for example: {[}100, 100, 0.5, 0.5{]} is a 50’ x 50’ flatskymap that has dimensions 100 x 100 with dx = dy = 0.5 arcminutes.

\end{itemize}

\sphinxlineitem{Returns}
\sphinxAtStartPar
\sphinxstylestrong{blsqinv} \textendash{} 1/Bl\textasciicircum{}2 either in 1d or 2D.

\sphinxlineitem{Return type}
\sphinxAtStartPar
array.

\end{description}\end{quote}

\end{fulllineitems}

\index{get\_nl() (in module tools)@\spxentry{get\_nl()}\spxextra{in module tools}}

\begin{fulllineitems}
\phantomsection\label{\detokenize{tools:tools.get_nl}}
\pysigstartsignatures
\pysiglinewithargsret{\sphinxcode{\sphinxupquote{tools.}}\sphinxbfcode{\sphinxupquote{get\_nl}}}{\sphinxparam{\DUrole{n}{noiseval\_in\_ukarcmin}}\sphinxparamcomma \sphinxparam{\DUrole{n}{el}}\sphinxparamcomma \sphinxparam{\DUrole{n}{beamval}\DUrole{o}{=}\DUrole{default_value}{None}}\sphinxparamcomma \sphinxparam{\DUrole{n}{elknee\_t}\DUrole{o}{=}\DUrole{default_value}{\sphinxhyphen{}1}}\sphinxparamcomma \sphinxparam{\DUrole{n}{alpha\_knee}\DUrole{o}{=}\DUrole{default_value}{0}}}{}
\pysigstopsignatures
\sphinxAtStartPar
Get the noise power spectra: White and 1/f spectrum.
Can return beam deconvolved nl if desired.
\begin{equation*}
\begin{split}P(f) = A^2 \left[ 1+ \left( \frac{\ell_{\rm knee}}{\ell}\right)^{\alpha_{\rm knee}} \right].\end{split}
\end{equation*}\begin{quote}\begin{description}
\sphinxlineitem{Parameters}\begin{itemize}
\item {} 
\sphinxAtStartPar
\sphinxstyleliteralstrong{\sphinxupquote{noiseval\_in\_ukarcmin}} (\sphinxstyleliteralemphasis{\sphinxupquote{float}}) \textendash{} White noise level in uK\sphinxhyphen{}arcmin.

\item {} 
\sphinxAtStartPar
\sphinxstyleliteralstrong{\sphinxupquote{el}} (\sphinxstyleliteralemphasis{\sphinxupquote{array}}) \textendash{} Multipoles over which the window function must be defined.

\item {} 
\sphinxAtStartPar
\sphinxstyleliteralstrong{\sphinxupquote{beamval}} (\sphinxstyleliteralemphasis{\sphinxupquote{float}}) \textendash{} Beam FWHM in arcmins.
Default is None.
If supplied, bl\textasciicircum{}2 will be divided from nl.

\item {} 
\sphinxAtStartPar
\sphinxstyleliteralstrong{\sphinxupquote{elknee\_t}} (\sphinxstyleliteralemphasis{\sphinxupquote{float}}) \textendash{} Knee frequency for 1/f  (el\_knee in the above equation).

\item {} 
\sphinxAtStartPar
\sphinxstyleliteralstrong{\sphinxupquote{alpha\_knee}} (\sphinxstyleliteralemphasis{\sphinxupquote{float}}) \textendash{} Slope for 1/f (alpha\_knee in the above equation).

\end{itemize}

\sphinxlineitem{Returns}
\sphinxAtStartPar
\sphinxstylestrong{nl} \textendash{} (Beam deconvoled) noise power spectrum.

\sphinxlineitem{Return type}
\sphinxAtStartPar
array.

\end{description}\end{quote}

\end{fulllineitems}



\renewcommand{\indexname}{Python Module Index}
\begin{sphinxtheindex}
\let\bigletter\sphinxstyleindexlettergroup
\bigletter{f}
\item\relax\sphinxstyleindexentry{flatsky}\sphinxstyleindexpageref{flatsky:\detokenize{module-flatsky}}
\indexspace
\bigletter{i}
\item\relax\sphinxstyleindexentry{inpaint}\sphinxstyleindexpageref{inpaint:\detokenize{module-inpaint}}
\indexspace
\bigletter{t}
\item\relax\sphinxstyleindexentry{tools}\sphinxstyleindexpageref{tools:\detokenize{module-tools}}
\end{sphinxtheindex}

\renewcommand{\indexname}{Index}
\printindex
\end{document}